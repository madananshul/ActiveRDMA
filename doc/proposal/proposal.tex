\documentclass[10pt]{article}
\usepackage{fullpage}
\usepackage[top=1in,bottom=1in,left=1in,right=1in]{geometry}

\title{15-712 Project: Active RDMA}
\author{Chris Fallin, Anshul Madan, Filipe Milit\~{a}o}
\date{}
\begin{document}

\maketitle

\section{Idea}

Traditional distributed services have built upon RPC or RPC-like
interfaces at the network level. In these systems, a server exports a
static set of operations, typically at a high level. For example, a
file server allows a client to open a file in one call, and read a
block of data in the next. An alternate paradigm exists: RDMA (Remote
Direct Memory Access), first proposed in~\cite{thekkath94}, exposes
data structures in the server's memory directly. Client applications
can then, with knowledge of the appropriate structures, implement most
intelligence on the client side and access shared data remotely. The
advantage of this scheme is that it allows low-latency operation if
the RDMA mechanism is implemented efficiently (i.e., in hardware or at
the low level in a system software trap). It also allows more
flexibility in client operations by allowing a finer granularity than
a static high-level API, although this aspect was not evaluated in the
original proposal. The disadvantage is that, for complex
data-structure manipulations or pathological cases (e.g., a long
linked list in memory), the cost of multiple round-trips over the
network can be extreme compared to performing the computation locally
at the server.

The proposal in~\cite{thekkath94} gets around this limitation by also
allowing RPC in some cases (e.g., when their nameserver implementation
cannot find an entry after a certain number of hash-table
probes). However, we see a better way. In the spirit of Active
Networks~\cite{AN-survey}, we propose to allow clients to send code to
the server to execute directly against the shared data structures. We
call this \textbf{Active RDMA}. This functionality will allow for the
\emph{flexibility} and \emph{client-driven intelligence} of RDMA while
providing the \emph{data locality} advantages of traditional
server-side code.

This proposal differs in several ways from previous work and other
alternatives:

\begin{itemize} \itemsep -2pt

\item Active Networks are primarily proposed for customization at the
  network or routing layer: for example, upgrading TCP~\cite{AN-tcp},
  or providing programmable packet-forwarding as in ANTS~\cite{ANTS},
  PLANet~\cite{planet} or FIRE~\cite{FIRE}.

\item Active Disks~\cite{AD2,AD} execute code at a disk controller,
  increasing data locality for certain types of data preprocessing in
  the same way as Active RDMA will allow for storage
  applications. However, Active Disks are limited in scope to this
  data processing. In contrast, Active RDMA is a generic mechanism
  that not only will allow for storage applications but any
  distributed system that requires some shared state at a central
  server.

\item Mobile code~\cite{mobile} is the name for a general research
  thrust of code migration over networks in order to build distributed
  systems. This is the most similar to our proposal. However, mobile
  code is generally a higher-level abstraction that does not provide
  direct access to a shared memory segment, for example, and provides
  more sandboxing and security that remove many of the potential
  performance benefits.

\end{itemize}

In order to narrow the scope of our proposal, we plan to focus
specifically on environments in which security is not an explicit
concern. This seems strange at first in the context of a
network-centric proposal, but such environments do exist. For example,
large computing clusters or datacenters with a single owner, in which
all machines or network devices are trusted and performance is the
paramount concern, are one possible application domain.

Furthermore, the assumption that security is externally provided
(whole-cluster access control) or otherwise an orthogonal problem
allows us to push Active RDMA's code execution to a lower level,
bypassing layers for higher performance. We plan to define our
interface to be sufficiently low-level that it could hypothetically
execute directly on a network interface card, interacting with shared
memory via DMA. This would allow for extremely low latency.

A secondary benefit to this, and one that we plan to explore, lies in
the flexibility of executing arbitrary code segments. A client could
dynamically generate this code; for example, a ``grep''-like operation
to search through some shared data structure could employ a
custom-compiled inner loop. If implemented in the context of a
distributed filesystem, this would combine the benefits of Active
Disk-like data locality for filtering with the flexibility of custom
code generation.

Our proposal thus encompasses these key points:

\begin{itemize} \itemsep -2pt

\item Defining a low-level ISA (instruction-set architecture) for
  remote code

\item Allowing clients to send remote code to a server to access
  shared memory directly

\item Demonstrating the capabilities of this architecture to: (i)
  implement a real distributed system (specifically, a filesystem);
  (ii) provide unique advantages by allowing arbitrary remote code
  execution.

\end{itemize}

\section{Work Plan}

We will implement Active RDMA in several stages, detailed below:

\subsection{Specifications}

We must first specify the interface that the Active RDMA mechanism
provides. This will encompass an ISA specification for the remote code
as well as the user-level APIs to (i) export memory on the server side
and (ii) send remote code to a server on the client side.

We plan to base our ISA specification on the low-level JVM bytecode
format, primarily in order to take advantage of existing
infrastructure and tools. There are several open-source JVMs from
which we can borrow a JIT engine, for example Kaffe.

\subsection{Execution Engine}

In order to make the prototype development feasible, we will not
initially pursue an optimized implementation of Active RDMA either in
a network interface card or in low-level system software. Rather, we
will prototype in userland. As mentioned above, we plan to base the
execution engine on a JVM core.

\subsection{Mobile Code Toolchain}

In order to use mobile code, clients need to be able to write it
efficiently. As a first pass, we can implement a simple assembler for
the bytecode format. It might be possible to adapt a simplified
compiler to this backend if we find ourselves with more time. However,
we expect that for our demonstration applications, mobile code
segments will either be relatively easy to construct by hand, or else
will be generated dynamically.

\subsection{Applications}

We will demonstrate Active RDMA's capabilities with a distributed
filesystem application. We can start by implementing basic
functionality to access files in a simple backing store, provided by
hooks on the server. Then, once that is complete, we can build a few
specific demonstrations of the flexibility of Active RDMA's mobile
code -- at least the canonical example of remote-grep, and perhaps
others.

\subsection{Distributed Filesystem Evaluation}

Given this distributed filesystem application, we will evaluate
performance against NFS and AFS. We will evaluate raw I/O performance,
and we will also develop benchmarks that take advantage of our unique
mobile-code features to showcase the strengths of our implementation.

\subsection{Additional Demos}

If time allows, we will develop additional (small) demonstration
applications that use mobile code in some way to implement some
distributed functionality.

\subsection{Goals}

As a baseline, 75\% goal, we hope to at least implement Active RDMA
and get a simple filesystem to work on top of it.

Our 100\% goal, and the one that we expect to achieve, is to build
specific mobile-code filesystem applications such as remote-grep and
demonstrate the strengths of our proposal.

If time allows and we wish to go much further, we will investigate (i)
implementing further applications that do interesting things with
mobile code, perhaps generating it dynamically, (ii) a more complete
toolchain for mobile code, and (iii) kernel-level implementations of
the Active RDMA execution engine.

\section{Resources}

We will require several network-connected machines to evaluate this
proposal. In development, we can use either a set of virtual machines
or our personal workstations. In the final evaluation, we may look
into finding a more complete, larger test network to stress-test the
implementation.

\section{Intended Outline}

\begin{itemize}
\item Introduction and RPC-RDMA Tradeoff
\item Related Work
\item Active RDMA
  \begin{itemize}
  \item ISA specifications
  \item several small examples of remote code segments (data structure manipulation)
  \end{itemize}
\item Active RDMA Filesystem (ARF)
  \begin{itemize}
    \item Shared data structures
    \item Server-side hooks
    \item Client-server partitioning tradeoff
    \end{itemize}
\item Dynamic Mobile Code: Remote Grep
  \begin{itemize}
  \item Mechanism: how to generate code
  \end{itemize}
\item Evaluation
\item Future Work
\end{itemize}

\bibliography{proposal}
\bibliographystyle{abbrv}

\end{document}
